
\chapter[Tópicos de Gerenciamento de Requisitos]{Tópicos de Gerenciamento de Requisitos}
\section{Rastreabilidade de Requisitos}
A Rastreabilidade de Requisitos (descrever, mostrar e acompanhar a vida de um requisito nos sentidos \emph{forward-to} e \emph{forward-from} \cite{garcia001}) é tido como uma das soluções para o problema da dessincronização entre \emph{software} e seus requisitos, sendo que a solução desse problema é quase que uma exigência básica da industria de desenvolvimento de \emph{software} atual \cite{leal001}. Faz parte de uma \emph{Specific Practice} da área de processo \emph{Manage Requirements}, do CMMI. É divida entre \emph{forward-to} (itens que derivam de um requisito \emph{x}) e \emph{forward-from} (itens que são derivados de um requisito \emph{x}).

Evidentemente que a Rastreabilidade de Requisitos muda de abordagem para abordagem. Será mostrada então o formato de rastreabilidade do grupo 1, levando em consideração que o framework de abordagem utilizado é o SAFe.

Primeiro, é lembrado que do tema de investimento será derivado o épico. Um épico pode ser classificado em arquitetural ou de negócio, e dele derivarão features. Das features serão derivadas estórias de usuário, que delas (as estórias de usuário) derivam tarefas e, no nosso contexto, testes funcionais.

A nomenclatura utilizada será uma concatenação entre prefixo (mostrados abaixos) e sufixos (número do índice do item). Lista de prefixos utilizados:

\begin{tabular}{c | c | c}
  \hline
  Item & Prefixo & Prefixo+sufixo (exemplo)\\ \hline
  Tema de Investimento & IT & IT0004 \\
  Epico & EP & EP0002 \\
  Feature & FE & FE0033 \\
  Estória de Usuário & US & US0243 \\
  Tarefas & TS & TS0932 \\
  Teste Funcional & FT & FT2142 \\
  \hline
\end{tabular}

\section{Atributos de Requisitos}
