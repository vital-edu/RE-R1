\chapter[Contexto]{Contexto}

O Ministério das Comunicações é uma instituição pública que tem como áreas de competência os serviços de radiodifusão, postais e de telecomunicações, sendo responsável por formular e propor as políticas nacionais para estas áreas \cite{mcsite001}. Possui a misão de desenvolver políticas públicas a fim de promover o acesso aos serviços de comunicações, contribuindo para o crescimento econômico, a inovação tecnológica e a inclusão social no Brasil \cite{mcsite001}.

Dentro do ministério há a Coordenação Geral da Tecnologia da Informação (CGTI) responsável por atender as demandas internas de desenvolvimento de software. As demandas são requisitadas através de diversos sistemas de \emph{software}. Sistemas esses que por serem utilizados no âmbito federal, tem um impacto direto para milhares de usuários, e indireto para milhões de cidadãos que necessitam que o órgao seja eficiente e eficaz em suas atividades, não prejudicando suas operações internas, e, consequentemente, externas.

Para tanto, a CGTI (mais precisamente a Divisão de Desenvolvimento de Sistemas - DSIS), necessita realizar a manutenção periódica dessas aplicações. Porém o processo de manutenção de \emph{software} é prejudicado pela dificuldade enfrentada tanto no gerenciamento quanto na execução das requisições que chegam à CGTI.

A grande quantidade de requisições contínuas tem evidenciado a falta de preparo da DSIS em conseguir atender essa demanda sem ter o seu fluxo de trabalho interrompido e prejudicado, sendo que, não raramente, algumas requisições acabam se perdendo, pela incapacidade de acompanharem de forma adequada o processo.

A manutenção das aplicações ocorre após haver o processo de gerenciamento de mudanças. Cada requisição é atrelada à uma categoria, que define a prioridade do processo:

\begin{itemize}
  \item Manutenção corretiva: eliminação de defeitos de códigos nos sistemas existentes;
  \item Manutenção adaptativa: adequações nas funcionalidades dos sistemas quando alteradas as regras e negócios, legislações e etc;
  \item Manutenção evolutiva: inserção de novas funcionalidades e/ou novos componentes nos sistemas existentes.
\end{itemize}

Há uma restrição quanto à quem pode requisitar as demandas, de acordo com a categoria atrelada ao processo:
\begin{itemize}
  \item Manutenção corretiva: qualquer usuário dos sistemas utilizados no MC;
  \item Manutenção adaptativa e evolutiva: apenas o gestor do sistema.
\end{itemize}

Dentro dessas categorias, o órgão necessita mapear o fluxo de manutenção, realizando:
\begin{itemize}
  \item A priorização das requisições;
  \item A execução das manutenções;
  \item O gerenciamento das execuções;
  \item A enrega ao cliente solicitante.
\end{itemize}

Sendo assim, o grupo 1, observando o contexto apresentado, deve apresentar uma solução que melhore o gerenciamento de requisições de manutenção de sistemas, dentro do Ministério das Comunicações.
