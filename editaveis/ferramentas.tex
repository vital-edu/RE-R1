\chapter[Ferramentas de Gerência de Requisitos]{Ferramentas de Gerência de Requisitos}

\emph{"If the process to select a RM tool seems daunting, the process to implement a tool is an even greater challenge"}\cite{beatty001}.

É mandatório a utilização de uma ferramenta de RM para times que desejam boa organização e qualidade no processo de desenvolvimento. Contudo, essa tarefa é difícil, e como dito na citação acima, mais difícil ainda é convencer um time a utilizá-la. Por tais motivos, pesquisou-se não somente por ferramentas completas: também analisou-se critérios como curva de aprendizagem, adaptação, customização, entre outros.

Assim sendo, foram inspecionadas mais a fundo três ferramentas de gerenciamento de requisitos: \emph{Rally}, \emph{Jama} e \emph{Polarion}. Devido a importância que o grupo dá a realização de um trabalho colaborativo e participativo, foi fator imprescindível a ferramenta ser \emph{web-based}, facilitando o acesso independente da plataforma de trabalho, ou lugar em que a equipe se encontre.

Embora a utilização de uma ferramenta instalável em um servidor pudesse ser uma opção alternativa que forneceria os mesmos benefícios descritos acima, decidiu-se que o esforço e tempo demandado para a instalação e configuração não traria benefícios suficientes, ainda mais considerando o contexto do trabalho.

Outro fator fundamental observado foi a compatibilidade com a metodologia adotada pela equipe.

Sendo assim, as ferramentas testadas serão apresentadas uma a uma, contemplando suas características, e, ao final da apresentação das ferramentas, será mostrada a decisão final com justificativa.

\section{Ferramenta 1 - Jama}
\begin{itemize}
  \item Suporte a múltiplos projetos
  \item Suporte a múltiplas equipes e papéis
  \item Importação de dados (doc/docx, xls/xlsx, xml, csv, arquivo de exportação do IBM Rational Doors)
  \item Impressão e exportação dos dados (doc, xls)
  \item Suporte nativo a: features, releases, sprints, épicos, estórias de usuário, pontos da estória, cenários de uso, e casos de teste
  \item Rastreabilidade dos requisitos
  \item Implementação nativa de alguns atributos de requisitos
  \item Integração com outros serviços
\end{itemize}

O Jama possui uma interface agradável, de baixa curva de aprendizagem, e com boas opções de gerenciamento de requisitos voltado para a abordagem adaptativa.

O padrão de criação de requisitos é baseado em campos de formulários e em modelos pré-definidos de documentos de texto que permitem uma flexibilização muito grande em sua criação, sendo bastante útil quando se quer seguir um padrão próprio ou alternativo na criação dos requisitos.

A ferramenta apresenta uma excelente capacidade de rastreabilidade relacionando os épicos com as estórias de usuários e com os casos de testes, funcionando em ambos os sentidos (para frente e para trás), e ainda possibilita a criação hierárquica dos requisitos, de forma extremamente eficiente.

São implementados os seguintes atributos de requisitos de forma nativa:

\begin{itemize}
  \item Fonte
  \item Prioridade
  \item Comentários
  \item Risco
  \item Prazo
\end{itemize}

Os demais atributos podem ser implementados por marcadores ou dentro da descrição do requisito, que permite a inserção de texto formatado.

A falta de alguns componentes gráficos como \emph{kanban} e \emph{backlog}, que seriam interessantes no contexto adaptativo fazem falta, mas podem ser facilmente substituídos por ferramentas que se integram ao Jama.

Uma interessante característica do Jama é permitir a criação de requisitos personalizados, o que possibilita a criação de temas de investimentos dentro da ferramenta.

A ferramenta apresenta algumas informações gerais sobre o projeto, dentre eles: risco da iteração e gráfico de priorização de estórias. E uma das grandes vantagens da ferramenta é o eficiente gerenciador de revisão, que permite comparar as edições realizadas pelos usuários nas estórias, épicos, e etc, permitindo a restauração dos documentos para uma versão anterior.

\section{Ferramenta 2 - Rally}
\begin{itemize}
  \item Suporte à múltiplos projetos
  \item Suporte à múltiplas equipes e papéis
  \item Interface prática, amigável e organizada
  \item Rastreabilidade dos requisitos
  \item Implementação nativa de alguns atributos de requisitos
  \item Diversas opções de estatísticas em forma de gráfico sobre o projeto
  \item Suporte nativo a: \emph{releases}, epicos, funcionalidades, estórias de usuário, iteração, pontos da estória, casos de teste, \emph{kanban} e \emph{backlog}
\end{itemize}

O Rally se destaca por sua interface funcional e prática, sendo bastante informativo e organizada. Pelo fato de possuir um kanban com cinco estágios, diversas opções gráficas para exibir o progresso do projeto, além de muitas informações relevantes para o gerenciamento do projeto e requisitos, a ferramenta se mostra poderosa e focada em projetos que usem abordagens adaptativas.

A rastreabilidade funciona muito bem quando se vai de épico  para estórias de usuários, mas o caminho contrário não é possível, pois há uma ligação apenas para os componentes filhos. No entanto isso pode ser contornado usando algumas das funções de visualização gráfica disponíveis na ferramenta.

São implementados os seguintes atributos de requisitos de forma nativa:

\begin{itemize}
  \item Comentários
  \item Status
  \item Prazo
  \item Nível de teste
  \item Esforço
\end{itemize}

Os demais atributos podem ser implementados por marcadores, mas a visualização desses marcadores não é tão prática, também podem ser implementados dentro da descrição do requisito, que permite a inserção de texto formatado.

Há apenas a opção de exportar os dados para arquivos PDFs, mas de forma individual, o que não se mostra muito útil. A possibilidade de se inserir arquivos anexos e a liberdade de criar estórias de usuário utilizando um editor de texto avançado permite uma flexibilização interessante das descrições das estórias.

Embora o gerenciador de revisão da ferramenta seja de fácil acesso, ela não tem funções importantes como comparação entre diferentes versões de um arquivo, e muito menos a opção de se restaurar o arquivo para uma versão anterior, o que o torna um tanto inútil, servindo apenas como log de alterações.

\section{Ferramenta 3 - Polarion}
\begin{itemize}
  \item Suporte a múltiplos projetos;
  \item Suporte a múltiplas equipes e papéis;
  \item Importação de dados (doc/docx e xls/xlsx)
  \item Exportação de dados (doc)
  \item Interface simples
  \item Rastreabilidade dos requisitos
  \item Algumas opções de estatistícas em forma gráfica sobre o projeto.
\end{itemize}

O Polarion se assemelha muito com um repositório de documentos editáveis, o que possui vantagens e desvantagens.

Como desvantagens, o sistema se distancia bastante de um sistema elegante e moderno de gerenciamento de requisitos, e não possui um suporte direto a metodologias ágeis, já que há nativamente apenas requisitos funcionais, requisitos do usuário,
casos de testes e \emph{features}.

A vantagem desse sistema é que tudo funciona em forma de \emph{templates}, ou seja, é possível personalizar os itens para assumirem a forma de épicos, mas tudo acontece como se estivesse produzindo um documento de texto. Mas não é por isso que a ferramenta deixa de apresentar recursos importantes como status, prioridade, rastreabilidade com ligações entre requisitos de forma bidirecional e histórico de versão, inclusive com opções de comparação entre as revisões, mas sem uma opção específica para a restauração do documento para uma versão anterior.

São implementados os seguintes atributos de requisitos de forma nativa:

\begin{itemize}
  \item Fonte
  \item Prazo
  \item Prioridade
  \item Comentários
\end{itemize}

Os demais atributos podem ser implementados dentro do corpo do requisito, que é um documento, onde praticamente tudo é passível de personalização. No entanto isso se mostra um processo complicado.

O Polarion é uma ferramenta versátil, que pode coibir os desavisados que ainda temem editar requisitos em forma de documentos de texto, mas basta dar uma olhada mais aprofundada para perceber que a ferramenta cumpre com os requisitos que qualquer gerenciador de requisitos deva possuir, possuindo uma personalização incrível, embora igualmente trabalhosa.

\section{Ferramenta Escolhida e Justificativa}

De todas as ferramentas testadas era esperado que o Rally fornecesse os componentes necessários para sua adoção como ferramenta de gerenciamento de requisitos, devido a sua forte integração com a abordagem adaptativa, no entanto, dois fatos considerados de grande importância acabaram tirando a ferramenta, inicialmente preferida, da disputa: a impossibilidade de rastrear bidirecionalmente os requisitos de forma fácil, e também a incapacidade de comparar e restaurar as modificações realizadas nos requisitos. E embora a ferrramenta apresente muitos atributos de requisitos implementados por /emph{default}, os demais atríbutos não são tão úteis quando implementados via marcadores, sendo difíceis de gerenciar.

A Polarion mostrou-se muito eficiente nas funções fundamentais de gerenciamento de requisitos, surpassando as expectativas e até mesmo mostrando-se mais indispensável que a ferramenta Rally. Seu alto poder de personalização é algo notável, mas essa customização é custosa e demasiadamente trabalhosa, tendo seus benefícios anulados. Ainda pesam contra a ferramenta: o fato de não ter ser nativamente adaptada aos métodos ágeis; ter uma forma de gerenciamento de requisitos considerada arcaica; e principalmente por não ter características interessantes e úteis presentes nas outras ferramentas. Por esses motivos também optou-se por não utilizá-la.

Sendo assim, optou-se pela ferramenta Jama. Poderia-se dizer que foi a alternativa restante, mas isso não faria jus ao que a ferramenta demonstrou. Sem dúvida, é a ferramenta mais completa entre as testadas, e embora não tenha uma interface tão agradável ou funcionalidades tão chamativas quanto as que são apresentadas na ferramenta Rally, possui o que é essencial, no que as demais ferramentas sempre deveriam fornecer: fácil e sólida rastreabilidade, com criação hierárquica de requisitos, e efetiva implementação não-nativa de atríbutos de requisitos.

A Jama, além de tudo é muito preparada para o contexto ágil, e possui integração nativa com ferramentas que extendem suas funcionalidades para outros contextos dentro do projeto de \emph{software}, não deixando a equipe presa à ferramenta para gerência de outras coisas.
