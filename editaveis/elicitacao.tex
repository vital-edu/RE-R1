\chapter[Técnicas de Elicitação de Requisitos]{Técnicas de Elicitação de Requisitos}
Descobrir requisitos é um dos grandes desafios do processo de produção de Software. Não ser capaz de completar essa tarefa da maneira certa e com qualidade impossibilitará o projeto de obter sucesso, não importa quão bom sejam as qualidades do time \cite[p. 227-228]{safe001}.

Por tal motivo, existe a necessidade de utilização de técnicas que ajudarão a levantar requisitos. Cada técnica apresenta melhor resultado em determinados contextos, e, muitas vezes, elas se completam. Por tal motivo, as técnicas foram escolhidas baseadas nas características do time.

As características que mais influenciaram na escolha das técnicas foram:
\begin{itemize}
  \item Disponibilidade: técnicas que exigem grande quantidade de encontros são mais difíceis de serem aplicadas pois os horários em que os times poderiam aplicá-las não é suficiente, embora seja flexível.
  \item Experiência: certas técnicas exigem um certo nível de experiência para apresentar resultados satisfatórios. O grupo não tem experiência com as técnicas ou as metodologias, logo, opta-se por técnicas mais fáceis.
\end{itemize}

\section{Workshop de Requisitos}

\section{Brainstorming}

\section{Entrevistas e Questionários}

\section{Mock-Ups}

\section{Product Council}

\section{Análise Competitiva}

\section{Customer change request system}

\section{Modelagem caso-de-uso}
