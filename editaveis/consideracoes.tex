\chapter[Considerações Finais]{Considerações Finais}
Ao fim da elaboração desse primeiro relatório, é possível constatar como o processo de Engenharia de Requisitos é essencial para a produção eficaz de \emph{software}. Processo esse tão importante que requisitos deficientes são a maior causa de falhas nos projetos de \emph{software} \cite{hofmann001}. Dessa forma concluimos que a Engenharia de Requisitos deve ser uma etapa do processo de desenvolvimento de \emph{software} que deve ser feita com um alto nível de qualidade, pois o custo de para se corrigir um erro aumenta significativamente ao longo do tempo \cite{boehm001}. Ou seja, quanto maior a qualidade e empenho da equipe nesta etapa do processo, menores serão os custos desnecessários, e maiores serão as chances de sucesso.


