\chapter[Planejamento do Projeto]{Planejamento do Projeto}
Neste capítulo será explicado o planejamento utilizado pelo Grupo 1, e justificativas sobre certas decisões.

\section{SAFe}
O SAFe traz tipos de planejamento diferentes para cada nível de projeto. Há o planejamento do épico, o planejamento da \emph{release}, o planejamento da \emph{sprint}, entre outros. Como já dito, o Nível de Portfolio foi retirado do contexto, ficando somente os níveis de Time e de Programa.

\subsection{Planejamento no Nível de Programa}
O planejamento no Nível de Programa será baseado no planejamento da \emph{release}, organizada pelo RTE no contexto do grupo 1. Além disso, para ajudar no planejamento, haverá o \emph{Roadmap}, que será atualizado sempre após o \emph{release planning}

\subsection{Planejamento no Nível de Time}
O planejamento no Nível de Time será baseado no planejamento da \emph{sprint}, organizada pelo Scrum Master no contexto do grupo 1. Haverá um \emph{kanban} da \emph{sprint} mostrando entregas a serem feitas, o que está sendo feito, etc. Após as \emph{sprints} ocorrerá a retrospectiva, e seus resultados irão influenciar no planejamento da próxima \emph{sprint}.

\section{Cronograma}
O cronograma ajudará as equipes fornecendo uma visão sobre as tarefas a serem desenvolvidas, o pré-requisito dessas tarefas, e o prazo disponível. Os cronogramas de MPR e Requisitos