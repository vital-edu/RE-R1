\chapter[Engenharia de Requisitos]{Engenharia de Requisitos}

\section{Scaled Agile Framework}
\subsection{Nível de Portfólio}
Responsável pelos artefatos \emph{epics} e \emph{investiment themes} \cite{safe005}, este nível de projeto traz os requisitos com maior nível de abstração entre os três níveis. Além disso, contém um tipo de time (\emph{portfolio management team}), tem seu próprio backlog (\emph{portfolio backlog}), e apresenta os conceitos de \emph{portfolio vision} e de \emph{architectural runway} \cite{safe005}.

\subsection{Nível de Programa}
Esse nível tem como principais objetivos manter a visão (Documento de Visão), gerenciar a \emph{release}, gerenciar a qualidade (integrando os resultados obtidos pelos times, e garantindo que os padrões de qualidade, performance, entre outros, estão sendo assegurados), fazer o \emph{deploy} do sistema, gerenciar recursos (ajustando prazo e gastos), e eliminar possíveis impedimentos (serão os facilitadores) \cite{safe002}.

\subsection{Nível de Time}
A unidade básica de trabalho para este nível é a \emph{estória de usuário}. O objetivo deste nível é definir, construir e testar as \emph{estórias de usuário} (com base nos \emph{critérios de aceitação}) no escopo da iteração, afim de se concluir mais partes do produto final \cite{safe001}.

\section{Papéis}
Um papel define comportamentos e responsabilidades de um indivíduo ou de um grupo de indivíduos que trabalham juntos como um time \cite{kruchten002}. O corportamento é expresso em termos de atividades que o papel pratica, e, cada papel é associado a um conjunto de atividades. No SAFe existem diferentes papéis para os diferentes níveis do sistema. No Nível de Portfólio vão haver, principalmente, papéis que irão interagir com os \emph{épicos} e \emph{temas de investimento}, no Nível de Programa, papéis que irão interagir com as \emph{features}, e, no Nível de Time, papéis que irão interagir com \emph{estórias de usuário}.

Nesta seção será dado uma breve definição de cada papel, e quem representa este papel no contexto do grupo 1.

\subsection{Nível de Portfólio}
\subsubsection{Epic Owner}
Responsável por definir e analisar o trabalho que será seguido \cite{safe007}.

\subsubsection{Enterprise Architect}


\subsection{Nível de Programa}

\subsubsection{Product Management}
\subsubsection{Release Management}
\subsubsection{System Team}
\subsubsection{Business Owners}
\subsubsection{RTE}
\subsubsection{System Architect}
\subsubsection{UX}
\subsubsection{Shared Resources}

\subsection{Nível de Time}
Neste nível somente aparecerão integrantes da parte de Requisitos. Os papeis interagem principalmente com as estórias de usuário, e não serão completamente fixos durante o projeto.

\subsubsection{Product Owner}
Responsável por representar o interesse de todos os envolvidos no projeto, faz parte do time e esta junto no dia-a-dia dos desenvolvedores e testadores, elaborando as estórias de usuário e ajudando o time a alcançar seus objetivos \cite{safe008}. No contexto do grupo 1, o \emph{product owner} irá flutuar durante o projeto (não será um papel estático).

\subsubsection{Scrum Master}
É responsável por facilitar o progresso do time (ajudando assim a se alcançar o objetivo final), liderar os esforços do time, reforçar as regras do processo ágil e eliminar impedimentos \cite{safe009}. No contexto do grupo 1, o papel de \emph{Scrum Master} irá flutuar durante o projeto, e provavelmente será escolhido um integrante do grupo de requisitos para ser \emph{Scrum Master} a cada sprint.

\subsubsection{Desenvolvedores e Testadores}
Responsáveis por desenvolverem e testarem as estórias de usuário. No contexto do grupo 1, todos os integrantes de Requisitos farão parte desse papel, sendo assim responsáveis pela elaboração da solução final.