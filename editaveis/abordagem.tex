\chapter[Escolha da Abordagem]{Escolha da Abordagem}
No desenvolvimento de um software é imprescindível a utilização de alguma abordagem que conduza seu processo produtivo.

Uma abordagem se refere a uma metodologia que será usada para estruturar, planejar, e controlar o processo de desenvolvimento do sistema \cite{CMS001}.

Logo, neste capítulo será justificada a escolha de uma determinada abordagem feita pelo grupo 1 da disciplina Requisitos de Software, responsável por realizar a manutenção de sistemas de software mantidos pelo Ministério das Comunicações. Primeiro será exposto uma visão geral de cada abordagem, explicando suas diferenças e os pontos positivos e negativos presentes em cada uma delas, para que enfim uma justificativa seja apresentada. As abordagens escolhidas para representar cada uma das correntes, tradicional e adaptativa, são o Rational Unified Process (RUP) e o Scaled Agile Framework (SAFe), respectivamente.

\section{Abordagem Tradicional - RUP}
O Processo Unificado surgiu em meados dos anos 90 e tem como proposta reunir diversas práticas e filosofias que se provaram eficientes até então no contexto de desenvolvimento de software \cite{kruchten001}.

Valoriza conceitos como arquitetura de software, desenvolvimento iterativo, gerência de requisitos, construção de modelos visuais para descrever o sistema, entre outros \cite{kruchten001}. A abordagem é baseada em disciplinas, onde cada disciplina agrupa uma série de atividades relacionadas e cada atividade produz diversos artefatos, que por sua vez contêm informações do processo e são submetidos à controle de versão (são produzidos, modificados e evoluídos durante o ciclo de vida do projeto). As disciplinas representam áreas de interesse presentes no processo de desenvolvimento de software \cite{kruchten001}.

O RUP estabelece a divisão do processo em quatro fases, sendo que cada fase possui um marco (objetivo principal) que será almejado durante sua execução. As fases por sua vez são divididas em iterações que consistem em um conjunto de atividades a serem realizadas para se obter um incremento do produto. Argumenta que o planejamento a longo prazo do projeto fornece uma visão à equipe de desenvolvimento do que deve ser feito e quando deve ser feito para que o projeto seja concluído e entregue no prazo. Analisando o processo unificado sob a perspectiva da disciplina de requisitos, é possível observar que o principal recurso para se representar requisitos dentro do processo são os casos de uso \cite{kruchten001}. Um caso de uso descreve uma interação entre uma entidade e o sistema e demonstra uma capacidade do software, assim como as funcionalidades que serão oferecidas ao usuário \cite{kruchten003}. Para atingir os objetivos da ER no RUP, a disciplina de requisitos descreve como definir uma visão do sistema, traduzir essa visão em um modelo de caso de uso (que, com especificações suplementares, definirá os requisitos detalhados do software), e como usar os atributos dos requisitos para ajudar a gerenciar o escopo do projeto e como mudar os requisitos do sistema \cite{kruchten004}.

%Apesar do RUP reagir bem a mudanças e especificar explicitamente que a mudança de requisitos é bem vinda, ele espera que até o início da terceira fase %do processo (geralmente 40\% do tempo total) 80\% dos requisitos do sistema já estejam bem definidos e estáveis.

Por fim, o RUP é um framework implementado por grandes empresas, mas determina que sua customização é aceita para que ele se adeque à realidade de organizações de menor porte, sendo possível definir as atividades e práticas do RUP que sejam do interesse da empresa em questão, desde que não sejam feridos os seus principais valores. As grandes empresas que o utilizam não o utilizam sempre da mesma maneira: enquanto algumas o utilizam de maneira formal, outras o customizam completamente \cite{kruchten005}.

\section{Abordagem Adaptativa - SAFe}
Bem mais novo, o Scaled Agile Framework (SAFe) traz ideias novas, mas tambem aspectos já consolidados na área de metodologias de software (Scrum, Kanban, etc).

Adaptável em diversos pontos, traz a ideia de dividir o projeto em três níveis: Portfolio, Program e Team. Cada nível produz seus artefatos, dispõe de diferentes papeis, tem responsabilidades, e interagem de maneira diferente \cite{safe003}.

Uma das características mais importantes em relação ao UP, é que, enquanto o UP é dirigido a UC, possuindo assim somente ele como descritor de comportamentos desejáveis para o sistema, no SAFe, existem três níveis de abstração para descrever tais comportamentos: o Epic, a Feature e a User Story \cite{safe004}. O Epic é bem mais alto nível, a Feature mais baixo nível que o Epic, e a User Story mais baixo nível que ambos \cite{safe004}.

\section{Escolha da Abordagem e Justificativa}
Levando em consideração os fatos levantados, é evidente que ambas as abordagens apresentam o que há de melhor disponível na escolha de Frameworks/Abordagens de produção de Software. O grupo então optou pela abordagem ágil utilizando o SAFe.

Os principais motivos foram o fato de ser uma abordagem bem mais nova (trazendo assim a oportunidade de se ter um contato inicial com algo completamente novo), e que ter mais níveis de projeto (Portfolio, Program e Team) e de descritores de comportamento (Epic, Feature e Story) parece ser mais interessante do que como essas coisas são lidadas no UP.
