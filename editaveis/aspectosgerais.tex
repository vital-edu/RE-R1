\chapter[Escolha da Abordagem]{Escolha da Abordagem}
No desenvolvimento de um software é imprescindível a utilização de alguma abordagem que conduza seu processo produtivo.

Uma abordagem se refere a uma metodologia que será usada para estruturar, planejar, e controlar o processo de desenvolvimento do sistema\cite{CMS001}

Logo, neste capítulo será justificada a escolha de uma determinada abordagem feita pelo grupo 1 da disciplina Requisitos de Software, responsável por realizar a manutenção de sistemas de software mantidos pelo Ministério das Comunicações. Primeiro será exposto uma visão geral de cada abordagem, explicando suas diferenças e os pontos positivos e negativos presentes em cada uma delas, para que enfim uma justificativa seja apresentada. As abordagens escolhidas para representar cada uma das correntes, preditiva e adaptativa, são o Rational Unified Process (RUP) e o Scaled Agile Framework (SAFe), respectivamente.

\section{Abordagem Tradicional - RUP}
O Processo Unificado surgiu em meados dos anos 90 e tem como proposta reunir diversas práticas e filosofias que se provaram eficientes até então no contexto de desenvolvimento de software. Mais tarde se tornou RUP (Rational Unified Process). Valoriza conceitos como arquitetura de software, desenvolvimento iterativo, gerência de requisitos, controlar mudanças, construção de modelos visuais para descrever o sistema, entre outros. Ele utiliza uma abordagem baseada em disciplinas, onde cada disciplina agrupa uma série de atividades relacionadas e cada atividade produz diversos artefatos, que por sua vez contêm informações do processo e são submetidos à controle de versão (são produzidos, modificados e evoluídos durante o ciclo de vida do projeto). As disciplinas representam áreas de interesse presentes à qualquer processo de desenvolvimento de software. O RUP estabelece a divisão do processo em quatro fases, sendo que cada fase possui um marco (objetivo principal) que será almejado durante sua execução. As fases por sua vez são divididas em iterações que consistem em um conjunto de atividades a serem realizadas para se obter um incremento do produto. O RUP preza pela previsibilidade do projeto, argumenta que o planejamento a longo prazo do projeto fornece uma visão à equipe de desenvolvimento do que deve ser feito e quando deve ser feito para que o projeto seja concluído e entregue no prazo. Analisando o processo unificado sob a perspectiva da disciplina de requisitos, é possível observar que o principal recurso para se representar requisitos dentro do processo são os casos de uso. Um caso de uso descreve uma interação entre uma entidade e o sistema, ele demonstra uma capacidade do software, as funcionalidades que serão oferecidas ao usuário. Dentro da disciplina de requisitos, existe um fluxo de atividades bem definido que procura estabelecer a compreensão dos requisitos entre todos os envolvidos no projeto (tanto organização-alvo quanto equipe de desenvolvimento), a definição e desenvolvimento dos requisitos e a verificação contínua da necessidade de mudanças. Apesar do RUP reagir bem a mudanças e especificar explicitamente que a mudança de requisitos é bem vinda, ele espera que até o início da terceira fase do processo (geralmente 40\% do tempo total) 80\% dos requisitos do sistema já estejam bem definidos e estáveis.
Por fim, o RUP foi um framework proposto para ser implementado por grandes empresas, mas determina que sua customização é aceita para que ele se adeque à realidade de organizações  de menor porte, sendo possível definir as atividades e práticas do RUP que sejam do interesse da empresa em questão, desde que não sejam feridos os seus principais valores.

\section{Abordagem Ágil}
\section{Escolha da Abordagem e Justificativa}

\chapter[Engenharia de Requisitos]{Engenharia de Requisitos}

\section{Scaled Agile Framework}
\section{Nível de Portfólio}
\section{Nível de Programa}
\section{Nível de Time}

\section{Papéis}
\section{Nível de Portfólio}
\section{Nível de Programa}
\section{Nível de Time}

\chapter[Elicitação de Requisitos]{Elicitação de Requisitos}
\section{Técnicas de Elicitação de Requisitos}

\chapter[Tópicos de Gerenciamento de Requisitos]{Tópicos de Gerenciamento de Requisitos}
\section{Rastreabilidade de Requisitos}
\section{Atributos de Requisitos}

\chapter[Planejamento do Projeto]{Planejamento do Projeto}
\section{Cronograma}

\chapter[Ferramentas de Gerência de Requisitos]{Ferramentas de Gerência de Requisitos}
\section{Ferramenta 1}
\section{Ferramenta 2}
\section{Ferramenta 3}
\section{Comparações}
\section{Definição}

