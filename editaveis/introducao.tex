\chapter[Introdução]{Introdução}

Este documento apresenta considerações gerais e preliminares relacionadas
à redação de relatórios de Projeto de Graduação da Faculdade UnB Gama
(FGA). São abordados os diferentes aspectos sobre a estrutura do trabalho,
uso de programas de auxilio a edição, tiragem de cópias, encadernação, etc.

\section{Visão Geral do Relatório}
O relatório será composto por capítulos, onde cada capítulo descreve e explica assuntos marcantes. Os capítulos são:

\subsection{Introdução}
Introdução do relatório, descreve brevemente como iremos abordar certos assuntos, qual o objetivo do trabalho e alguns conceitos.
\subsection{Contexto}
Descreve de maneira geral o contexto da Empresa e do problema a ser solucionado.
\subsection{Escolha da Abordagem}
Descreve um pouco sobre as duas abordagens (adaptativa e tradicional), justifica e faz uma escolha de abordagem.
\subsection{Modelos de Maturidade}
Fala um pouco sobre a definição de Modelos de Maturidade, e como eles foram utilizados no contexto do projeto.
\subsection{Engenharia de Requisitos}
Mostra todo o processo de Engenharia de Requisitos a ser utilizado no projeto (papéis, atividades, entre outras coisas).
\subsection{Planejamento do projeto}
Diz de forma geral como será dado o planejamento nos diferentes níveis do projeto e mostra o andamento do cronograma
\subsection{Técnicas de Elicitação de Requisitos}
Fala brevemente sobre diferentes técnicas de requisitos, quais vão e não vão ser utilizadas e o motivo desta escolha.
\subsection{Tópicos de Gerenciamento de Requisitos}
Fala como será dado a gerência de requisitos no contexto do trabalho, quais atributos serão utilizados para os requisitos, e como será a rastreabilidade.
\subsection{Ferramentas de Gerência de Requisitos}
Mostra um \emph{overview} sobre três ferramentas, a escolha de uma delas, e a justificativa desta escolha.
\subsection{Considerações Finais}
Considerações finais sobre o trabalho em si, e as conclusões obtidas.
\subsection{Referências}
Traz a fonte de algumas informações utilizadas como base para o relatório.
